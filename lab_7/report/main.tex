% chktex-file 8 36 26 38 32
\documentclass[a4paper,12pt]{article}

% Packages
\usepackage[utf8]{inputenc}
\usepackage{graphicx}
\usepackage{hyperref}
\usepackage{geometry}
\usepackage{float}
\geometry{margin=1in}
\usepackage{titlesec}
\usepackage{fancyhdr}
\usepackage{listings}
\usepackage{longtable}
\usepackage{booktabs}
\usepackage{xcolor}
\usepackage{minted}
\usepackage{colortbl}
\usepackage{setspace}
\usepackage[utf8]{inputenc}
\usepackage{newunicodechar}
\usepackage{amssymb}
\usepackage{lmodern} 
\usepackage{subcaption}
\usepackage{silence}
\WarningFilter{fancyhdr}{\headheight is too small}

\newunicodechar{✓}{\checkmark}  
\usepackage[most]{tcolorbox}
\tcbuselibrary{minted, listingsutf8}
\setminted{breaklines}

\newtcblisting{code}[2][]{%
  listing engine=minted,
  colback=white,
  colframe=black,
  listing only,
  minted language=#2,
  minted options={fontsize=\small, linenos, breaklines,
                  breaksymbolleft=, breaksymbolright=, #1},
  sharp corners,
  boxrule=0.8pt,
  left=5pt,
  right=5pt,
  top=8pt,
  bottom=8pt,
  enhanced,
  breakable
}


\lstset{
  basicstyle=\ttfamily\small,
  keywordstyle=\color{blue},
  stringstyle=\color{red},
  commentstyle=\color{green!50!black},
  breaklines=true,
  frame=single,
  numbers=left,
  numberstyle=\tiny,
  stepnumber=1,
  numbersep=5pt,
  showstringspaces=false,
  tabsize=2,
}

% Header and footer setup for all pages except cover
\fancypagestyle{main}{
  \fancyhf{}
  \rhead{CSC381 - E-Commerce}
  \lhead{Lab 7: Shopping Cart Implementation}
  \cfoot{\thepage}
  \renewcommand{\headrulewidth}{0.4pt}
  \renewcommand{\footrulewidth}{0.4pt}
}

\title{Lab 7: Shopping Cart Implementation}
\author{Submitted by: Parakram Kharel \\ Roll No: 24 \\ 
\textit{Kathford International College of Engineering and Management} \\ 
Affiliated to Tribhuvan University}
% \date{\today}
\makeatletter
\begin{document}
% \date{August 17, 2025}

% --------- Cover Page ---------
\begin{titlepage}
  \begin{center}
    \vspace*{2cm}
    \includegraphics[width=0.35\textwidth]{Kath.png} \\[2cm]

    {\Huge \bfseries Lab Report 7} \\[0.5cm]
    {\Large Shopping Cart Implementation} \\[2cm]

    {\Large \textbf{Course:} CSC381 - E-Commerce} \\[1cm]
    {\Large \textbf{Submitted by:}} \\[0.3cm]
    {\large Parakram Kharel \\ Roll No: 24} \\[2cm]

    \textit{Kathford International College of Engineering and Management} \\
    Affiliated to Tribhuvan University \\[3cm]
    
    \date{August 24, 2025}
    {\normalsize \@date}
    % \today
  \end{center}
\end{titlepage}

% Use fancy header/footer for the rest of the document
\pagestyle{main}
\setlength{\parskip}{1em}

% --------- Content ---------
\section*{1. Objective}
To implement a shopping cart in PHP that allows users to add, remove, and update items dynamically using sessions.

\section*{2. Tools and Technologies Used}
\begin{longtable}{ll}
\toprule
\textbf{Technology} & \textbf{Purpose} \\
\midrule
HTML5 & Cart pages and form structure \\
CSS3 & Layout and visual styling \\
JavaScript & Inline confirmations and UI behavior \\
PHP & Server-side logic, CRUD, Sessions \\
MySQL & Persistent storage for catalog and orders \\
Caddy & Web server and local hosting \\
Bootstrap 5 & Admin UI, tables, forms, responsive grid \\
VS Code & Development environment \\
Browser DevTools & Testing and debugging \\
\bottomrule
\end{longtable}

\section*{3. Theory / Background}
A shopping cart is a temporary online storage area that keeps track of products selected by the user before checkout.
In PHP-based systems, the cart is usually implemented using sessions, which store user data temporarily on the server.
When a user adds an item to the cart:
\begin{itemize}
\item PHP creates a session for that user.
\item Product details (name, price, quantity, etc.) are stored in the session array.
\item The cart data persists across multiple pages until the user checks out or logs out.
\end{itemize}

\section*{4. Page Layout Design}
\subsection*{4.1 Cart Format}
\begin{itemize}
  \item Product image, name, price
  \item Quantity input field
  \item "Remove” and “Update” buttons
\end{itemize}

\section*{5. Code Snippets}

\subsection*{5.1 Database Connection}
\begin{code}[fontsize=\small]{php}
<?php
$host = "localhost";
$user = "root";
$pass = "root";
$dbname = "ecom";

$conn = mysqli_connect($host, $user, $pass, $dbname);

if (!$conn) {
    die("Connection failed: " . mysqli_connect_error());
}
?>

\end{code}

\subsection*{5.2 Cart Add}
\begin{code}[fontsize=\small]{php}
<?php
session_start();

if (isset($_POST['add_to_cart'])) {
    $product_id = $_POST['product_id'];
    $product_name = $_POST['product_name'];
    $product_price = $_POST['product_price'];
    $quantity = $_POST['quantity'];

    $cart_item = array(
        'id' => $product_id,
        'name' => $product_name,
        'price' => $product_price,
        'quantity' => $quantity
    );

    if (!isset($_SESSION['cart'])) {
        $_SESSION['cart'] = array();
    }

    $_SESSION['cart'][$product_id] = $cart_item;
    header("Location: cart.php");
    exit();
}
?>
\end{code}

\subsection*{5.3 Update / Clear Cart}
\begin{code}[fontsize=\small]{php}
    <?php
// Remove item
if (isset($_GET['remove'])) {
    $id = $_GET['remove'];
    unset($_SESSION['cart'][$id]);
    header("Location: cart.php");
}

// Update quantity
if (isset($_POST['update_cart'])) {
    foreach ($_POST['quantity'] as $id => $qty) {
        $_SESSION['cart'][$id]['quantity'] = $qty;
    }
    header("Location: cart.php");
}
?>
\end{code}

\subsection*{5.4 PHP: Display Cart Items}
\begin{code}[fontsize=\small]{php}
<?php
  session_start();
  $total = 0;
?>

<h2>Your Shopping Cart</h2>
<table border="1" cellpadding="10">
<tr>
    <th>Product</th>
    <th>Price</th>
    <th>Quantity</th>
    <th>Total</th>
    <th>Action</th>
</tr>

<?php
if (!empty($_SESSION['cart'])) {
    foreach ($_SESSION['cart'] as $item) {
        $subtotal = $item['price'] * $item['quantity'];
    $total += $subtotal;
    echo "
    <tr>
        <td>{$item['name']}</td>
        <td>\${$item['price']}</td>
        <td>{$item['quantity']}</td>
        <td>\${$subtotal}</td>
        <td><a href='cart.php?remove={$item['id']}'>Remove</a></td>
    </tr>";
    }
    echo "<tr><td colspan='3'><b>Grand Total</b></td><td>\$$total</td><td></td></tr>";
} else {
    echo "<tr><td colspan='5'>Your cart is empty!</td></tr>";
}
?>
</table>

\end{code}

\section*{6. Output / Screenshots}

\begin{figure}[H]
  \centering
  \includegraphics[width=\textwidth]{images/products_with_add_cart.png}
  \caption{Products with Add to Cart button.}
  \label{fig:img1}
\end{figure}

\begin{figure}[H]
  \centering
  \includegraphics[width=\textwidth]{images/add_cart_toast.png}
  \caption{Add to Cart Toast.}
  \label{fig:img2}
\end{figure}

\begin{figure}[H]
  \centering
  \includegraphics[width=\textwidth]{images/cart.png}
  \caption{Shopping Cart.}
  \label{fig:img3}
\end{figure}

\begin{figure}[H]
  \centering
  \includegraphics[width=\textwidth]{images/clear_cart_confirm.png}
  \caption{Clear Cart Confirmation.}
  \label{fig:img4}
\end{figure}

\begin{figure}[H]
  \centering
  \includegraphics[width=\textwidth,height=5cm]{images/empty_cart.png}
  \caption{Empty Cart.}
  \label{fig:img5}
\end{figure}

\section*{7. Result}
User can add, update, or remove items from the cart.The total quantity and amount update dynamically.Cart data remains stored in the session until the user closes the browser or checks out.

\section*{8. Conclusion}
This lab demonstrated how to use PHP sessions to build a shopping cart. It improved understanding of session management, dynamic page updates, and e-commerce functionality.

\section*{9. References}
\begin{itemize}
  \item \href{https://www.php.net/manual/en/mysqli.prepare.php}{PHP mysqli Prepared Statements}
  \item \href{https://www.php.net/manual/en/function.htmlspecialchars.php}{PHP htmlspecialchars}
  \item \href{https://www.php.net/manual/en/session.examples.basic.php}{PHP Session Examples}
  \item \href{https://getbootstrap.com/docs/5.3/getting-started/introduction/}{Bootstrap 5 Documentation}
  \item \href{https://developer.mozilla.org/en-US/docs/Web/HTTP}{MDN HTTP Reference}
  \item \href{https://caddyserver.com/docs/}{Caddy Web Server Documentation}
\end{itemize}

\end{document}
